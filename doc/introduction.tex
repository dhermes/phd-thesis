\chapter{Introduction}

This is a work in two parts, each in a different subfield of mathematics.
The first part is a general-purpose tool for computational physics problems.
Since the tool requires a significant amount of computational geometry, the
second half focuses on computational geometry. In particular, it considers
cases where the geometric methods used have seriously degraded accuracy due to
ill-conditioning.

\section{Computational Physics}

In computational physics, the problem of data transfer between meshes
occurs in several applications. For example, by allowing the underyling
computational domain to change during a simulation, computational
effort can be focused dynamically to resolve sensitive features
of a numerical solution. Mesh adaptivity (see, for example,
\cite{Dukowicz1987, Babuska1978}), this in-flight change in the mesh,
requires translating the numerical solution from the old mesh to the new,
i.e. data transfer. As another example, Lagrangian or particle-based methods
treat each node in mesh as a particle and so with each timestep the mesh
travels \textbf{with} the fluid (see, for example, \cite{Hirt1974}).
However, over (typically limited) time the mesh
becomes distorted and suffers a loss in element quality which causes
catastrophic loss in the accuracy of computation. To overcome this, the
domain must be remeshed or rezoned and the solution must be
transferred (remapped) onto the new mesh configuration.

When pointwise interpolation is used to transfer a solution, quantities with
physical meaning (e.g. mass, concentration, energy) may not be conserved.
To address this, there have been many explorations (for example,
\cite{Jiao2004, Farrell2009, Farrell2011}) of
\textbf{conservative interpolation} (typically using Galerkin or
\(L_2\)-minimizing methods). In this work, the author introduces a
conservative interpolation method for data transfer between high-order
meshes. These high-order meshes are typically curved, but not necessarily
all elements or at all timesteps.

High-order CFD is good: \cite{Wang2013}.
%% High-order CFD methods have received considerable attention in the
%% research community in the past two decades because of their potential
%% in delivering higher accuracy with lower cost than low-order
%% methods
%% ...
%% Amazingly, we received a unanimous definition of high order:
%% third order or higher

%% Many practical problems involve time-varying geometries,
%% such as rotor-stator flows, flapping flight or
%% fluid-structure interactions. In such cases, it is necessary to properly
%% account for the time variation of the solution domain if accurate solutions
%% are to be obtained.

%% https://doi.org/10.1016/j.cma.2009.01.012 (Per and Peraire)
%% There is a growing interest in high-order methods for fluid
%% problems, largely because of their ability to produce highly accurate
%% solutions with minimum numerical dispersion.

%% J. Donea, Arbitrary Lagrangian-Eulerian finite element methods, in: T.
%% Belytschko, T. Hughes (Eds.), Computational Methods for Transient Analysis,
%% vol. 1, Elsevier, Amsterdam, 1983, pp. 474-515.
%% I. Lomtev, R.M. Kirby, G.E. Karniadakis, A discontinuous Galerkin ALE method
%% for compressible viscous flows in moving domains, J. Comput. Phys. 155 (1)
%% (1999) 128-159.

\section{Computational Geometry}

Placeholder.
