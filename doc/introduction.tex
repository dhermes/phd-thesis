\chapter{Introduction}

This is a work in two parts, each in a different subfield of mathematics.
The first part is a general-purpose tool for computational physics problems.
The tool enables data transfer across two curved meshes.
Since the tool requires a significant amount of computational geometry, the
second half focuses on computational geometry. In particular, it considers
cases where the geometric methods used have seriously degraded accuracy due to
ill-conditioning.

\section{Computational Physics}

In computational physics, the problem of data transfer between meshes
occurs in several applications. For example, by allowing the underyling
computational domain to change during a simulation, computational
effort can be focused dynamically to resolve sensitive features
of a numerical solution. Mesh adaptivity (see, for example,
\cite{Dukowicz1987, Babuska1978}), this in-flight change in the mesh,
requires translating the numerical solution from the old mesh to the new,
i.e. data transfer. As another example, Lagrangian or particle-based methods
treat each node in mesh as a particle and so with each timestep the mesh
travels \textbf{with} the fluid (see, for example, \cite{Hirt1974}).
However, over (typically limited) time the mesh
becomes distorted and suffers a loss in element quality which causes
catastrophic loss in the accuracy of computation. To overcome this, the
domain must be remeshed or rezoned and the solution must be
transferred (remapped) onto the new mesh configuration.

When pointwise interpolation is used to transfer a solution, quantities with
physical meaning (e.g. mass, concentration, energy) may not be conserved.
To address this, there have been many explorations (for example,
\cite{Jiao2004, Farrell2009, Farrell2011}) of
\textbf{conservative interpolation} (typically using Galerkin or
\(L_2\)-minimizing methods). In this work, the author introduces a
conservative interpolation method for data transfer between high-order
meshes. These high-order meshes are typically curved, but not necessarily
all elements or at all timesteps.

The existing work on data transfer has considered straight-sided meshes,
which use shape functions that have degree \(p = 1\) to represent solutions
on each element. However, both to allow for greater geometric flexibility
and for high-order of convergence, this paper will consider the case
of curved isoparametric\footnote{I.e. the degree of the discrete field on the
mesh is same as the degree of the shape functions that determine the
mesh.} meshes. Allowing curved geometries is useful since many practical
problems involve geometries that change over time, such as flapping flight
or fluid-structure interactions. In addition, high-order CFD methods
(\cite{Wang2013}) have the ability to produce highly accurate solutions
with low dissipation and low dispersion error.

\section{Computational Geometry}

As a parameter approaches a root of a function in Bernstein form,
the condition number of evaluation becomes infinite. Similarly,
as a (transversal) intersection of two B\'{e}zier curves approaches a
point of tangency, the condition number of intersection becomes
infinite. In either case...

``double roots, though rare, are overwhelmingly more common in practice
than are roots of higher multiplicity'' (\cite{Kahan1972}, page 6)

\section{Organization}

This work is organized as follows. Chapter~\ref{chap:preliminaries}
establishes common notation and reviews basic results relevant to the
topics at hand. Chapter~\ref{chap:bezier-intersection} is an
in-depth discussion of the computational geometry methods we'll need
to implement to enable data transfer. Chapter~\ref{chap:data-transfer}
describes the data transfer process and gives results of some
numerical experiments confirming the rate of convergence.
Chapter~\ref{chap:k-compensated} describes a compensated algorithm for
evaluating functions in Bernstein form (such as B\'{e}zier curves);
this algorithm produces results that are as accurate as if
the computations were done in \(K\) times the working precision
and then rounded back to the working precision.
Chapter~\ref{chap:compensated-newton} describes two modified Newton's
methods which allow for greater accuracy in the presence of
ill-conditioning; one is used for computing simple zeros
of polynomials in Bernstein form and the other for computing
B\'{e}zier curve intersections.
