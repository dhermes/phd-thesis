\chapter{Proof Details}\label{chap:appendix-proof-details}

\begin{proof}[Proof of Lemma~\ref{lemma:ell-tilde}]\label{proof:ell-tilde}
We'll start with the \(F = 1\) case. Recall where the terms originate:
\begin{align}
\left[P_1, e_1\right] &= \mathtt{TwoProd}\left(\widehat{r},
  \widehat{b}_j^{(k + 1)}\right) \\
\left[P_2, e_2\right] &= \mathtt{TwoProd}\left(s,
  \widehat{b}_{j + 1}^{(k + 1)}\right) \\
\left[\widehat{b}_j^{(k)}, e_3\right] &= \mathtt{TwoSum}\left(P_1, P_2\right).
\end{align}
Hence Theorem~\ref{thm:eft} tells us that
\begin{align}
\left|P_1\right| &\leq (1 + \mach)\left|\widehat{r} \cdot
  \widehat{b}_j^{(k + 1)}\right| \leq (1 + \mach)^2 (1 - s)
  \left|\widehat{b}_j^{(k + 1)}\right| \\
\left|e_1\right| &\leq \mach \left|\widehat{r} \cdot
  \widehat{b}_j^{(k + 1)}\right| \leq \mach(1 + \mach)(1 - s) \left|
  \widehat{b}_j^{(k + 1)}\right| \\
\left|P_2\right| &\leq (1 + \mach) s \left|\widehat{b}_{j + 1}^{(k + 1)}\right| \\
\left|e_2\right| &\leq \mach s \left|\widehat{b}_{j + 1}^{(k + 1)}\right| \\
\left|e_3\right| &\leq \mach \left|P_1\right| + \mach\left|P_2\right| \\
\left|\rho \cdot \widehat{b}_j^{(k + 1)}\right| &\leq
(1 + \mach)(1 - s) \left|\widehat{b}_j^{(k + 1)}\right|.
\end{align}
In general, we can swap \(\mach\left|P_j\right|\) for
\((1 + \mach)\left|e_j\right|\) based on how closely related the bound
on the result and the bound on the error are. Thus
\begin{align}
\widetilde{\ell}_{1, j}^{(k)} &= \left|e_1\right| + \left|e_2\right| +
  \left|e_3\right| + \left|\rho \cdot \widehat{b}_j^{(k + 1)}\right| \\
&\leq (2 + \mach)\left(\left|e_1\right| + \left|e_2\right|\right) +
  (1 + \mach)(1 - s) \left|\widehat{b}_j^{(k + 1)}\right| \\
&\leq \left[(1 + \mach)^3 - 1\right] (1 - s) \left|
  \widehat{b}_j^{(k + 1)}\right| + \left[(1 + \mach)^2 - 1\right] s \left|
  \widehat{b}_{j + 1}^{(k + 1)}\right| \\
&\leq \gamma_3 \left((1 - s) \left|\widehat{b}_j^{(k + 1)}\right| +
  s \left|\widehat{b}_{j + 1}^{(k + 1)}\right|\right).
\end{align}
For \(\widetilde{\ell}_{F + 1}\), we want to relate the ``current'' errors
\(e_1, \ldots, e_{5F + 3}\) to the ``previous'' errors \(e_1',
\ldots, e_{5F - 2}'\) that show up in \(\widetilde{\ell}_F\). In the same
fashion as above, we track where the current errors come from:
\begin{align}
\left[S_1, e_1\right] &= \mathtt{TwoSum}\left(e_1', e_2'\right) \\
\left[S_2, e_2\right] &= \mathtt{TwoSum}\left(S_1, e_3'\right) \\
&\mathrel{\makebox[\widthof{=}]{\vdots}} \nonumber \\
\left[S_{5F - 3}, e_{5F - 3}\right] &=
  \mathtt{TwoSum}\left(S_{5F - 4}, e_{5F - 2}'\right) \\
\left[P_{5F - 2}, e_{5F - 2}\right] &= \mathtt{TwoProd}\left(\rho,
  \cdb{F - 1}_j^{(k + 1)}\right) \\
\left[\widehat{\ell}_{F, j}^{(k)}, e_{5F - 1}\right] &=
  \mathtt{TwoSum}\left(S_{5F - 3}, P_{5F - 2}\right) \\
\left[P_{5F}, e_{5F}\right] &= \mathtt{TwoProd}\left(s,
  \cdb{F}_{j + 1}^{(k + 1)}\right) \\
\left[S_{5F + 1}, e_{5F + 1}\right] &=
  \mathtt{TwoSum}\left(\widehat{\ell}_{F, j}^{(k)}, P_{5F}\right) \\
\left[P_{5F + 2}, e_{5F + 2}\right] &= \mathtt{TwoProd}\left(\rho,
  \cdb{F}_j^{(k + 1)}\right) \\
\left[\cdb{F}_j^{(k)}, e_{5F + 3}\right] &= \mathtt{TwoSum}\left(
  S_{5F + 1}, P_{5F + 2}\right).
\end{align}
Arguing as we did above, we start with
\(\left|e_1\right| \leq \mach \left|e_1'\right| + \mach \left|e_2'\right|\)
and build each bound recursively based on the previous, e.g.
\(\left|e_2\right| \leq \mach \left|S_1\right| + \mach \left|e_3'\right| \leq
(1 + \mach) \mach \left|e_1'\right| + (1 + \mach) \mach \left|e_2'\right| +
\mach \left|e_3'\right|\). Proceeding in this fashion, we find
\begin{align}
\widetilde{\ell}_{F + 1, j}^{(k)} &= \left|e_1\right| + \cdots +
  \left|e_{5F + 3}\right| + \left|\rho \cdot \cdb{F}_j^{(k + 1)}\right| \\
&\leq \gamma_{5F} \left|e_1'\right| + \gamma_{5F} \left|e_2'\right| +
  \gamma_{5F - 1} \left|e_3'\right| + \cdots +
  \gamma_4 \left|e_{5F - 2}'\right| +
  \gamma_4 \left|\rho \cdot \cdb{F - 1}_j^{(k + 1)}\right| \\
&\qquad + \gamma_3 (1 - s) \left|
  \cdb{F}_j^{(k + 1)}\right| + \gamma_3 s \left|
  \cdb{F}_{j + 1}^{(k + 1)}\right| \\
&\leq \gamma_3 \left(
  (1 - s) \left|\cdb{F}_j^{(k + 1)}\right| +
  s \left|\cdb{F}_{j + 1}^{(k + 1)}\right|\right) +
  \gamma_{5F} \cdot \widetilde{\ell}_{F, j}^{(k)}
\end{align}
as desired.
\end{proof}

\begin{proof}[Proof of Lemma~\ref{lemma:L-and-D-bounds}]\label{proof:L-and-D-bounds}
First, note that for \textbf{any} sequence \(v_0, \ldots, v_{k + 1}\) we
must have
\begin{equation}
\sum_{j = 0}^k \left[(1 - s) v_j + s v_{j + 1}\right] B_{j, k}(s) =
\sum_{j = 0}^{k + 1} v_j B_{j, k + 1}(s).
\end{equation}
For example of this in use, via \eqref{eq:ell-tilde-1}, we have
\begin{equation}
  L_{1, k} \leq \gamma_3 \sum_{j = 0}^{k + 1} \left|
  \widehat{b}_j^{(k + 1)}\right| B_{j, k + 1}(s).
\end{equation}
In order to work with sums of this form, we define Bernstein-type
sums related to \(L_{F, k}\):
\begin{align}
D_{0, k} &\coloneqq \sum_{j = 0}^k \left|\widehat{b}_j^{(k)}\right|
B_{j, k}(s) \\
D_{F, k} &\coloneqq \sum_{j = 0}^k \left|\cdb{F}_j^{(k)}\right| B_{j, k}(s).
\end{align}
Hence Lemma~\ref{lemma:ell-tilde} gives
\begin{align}
L_{1, k} &\leq \gamma_3 D_{0, k + 1} \label{eq:ell-1-k} \\
L_{F + 1, k} &\leq \gamma_3 D_{F, k + 1} + \gamma_{5F} L_{F, k}
\label{eq:ell-F-k}
\end{align}
In addition, for \(F \geq 1\) since
\begin{align}
\cdb{F}_j^{(k)} &= \widehat{\ell}_{F, j}^{(k)} \oplus \left(
  s \otimes \cdb{F}_{j + 1}^{(k + 1)}\right) \oplus \left((1 \ominus s) \otimes
  \cdb{F}_{j}^{(k + 1)}\right) \\
&= (1 - s) \cdot \cdb{F}_j^{(k + 1)}(1 + \theta_3) +
  s \cdot \cdb{F}_{j + 1}^{(k + 1)}(1 + \theta_3) +
  \widehat{\ell}_{F, j}^{(k)} (1 + \theta_2)
\end{align}
we have
\begin{equation}\label{eq:df-first}
D_{F, k} \leq (1 + \gamma_3) D_{F, k + 1} + (1 + \gamma_2) \sum_{j = 0}^k
\left|\widehat{\ell}_{F, j}^{(k)}\right| B_{j, k}(s).
\end{equation}
Since \(\ell_{F, j}^{(k)}\) has \(5F - 1\) terms (only the last of which
involves a product), the terms in the computed value will be involved in
at most \(5F - 2\) flops, hence
\(\left|\widehat{\ell}_{F, j}^{(k)}\right| \leq
\left(1 + \gamma_{5F - 2}\right) \widetilde{\ell}_{F, j}^{(k)}.\)
Combined with \eqref{eq:df-first} and the fact that there is no local error
when \(F = 0\), this means
\begin{align}
D_{0, k} &\leq (1 + \gamma_3) D_{0, k + 1} \label{eq:d-0-k} \\
D_{F, k} &\leq (1 + \gamma_3) D_{F, k + 1} + (1 + \gamma_{5F}) L_{F, k}.
\label{eq:d-F-k}
\end{align}
The four inequalities \eqref{eq:ell-1-k}, \eqref{eq:ell-F-k}, \eqref{eq:d-0-k}
and \eqref{eq:d-F-k} allow us to write all bounds in terms of
\(D_{0, n} = \widetilde{p}(s)\) and \(D_{F, n} = 0\). From \eqref{eq:d-0-k}
we can conclude that \(D_{0, n - k} \leq \left(1 + \gamma_{3k}\right) \cdot
\widetilde{p}(s)\) and from \eqref{eq:ell-1-k} that \(L_{1, n - k} \leq
\gamma_3 \left(1 + \gamma_{3(k - 1)}\right) \cdot \widetilde{p}(s)\).

To show the bounds for higher values of \(F\), we'll assume we have
bounds of the form
\(D_{F, n - k} \leq \left(q_F(k) \mach^F + \bigO{\mach^{F + 1}}\right) \cdot
\widetilde{p}(s)\) and
\(L_{F, n - k} \leq \left(r_F(k) \mach^F + \bigO{\mach^{F + 1}}\right) \cdot
\widetilde{p}(s)\) for two families of polynomials \(q_F(k), r_F(k)\). We
have \(q_0(k) = 1\) and \(r_1(k) = 3\) as our base cases and can build from
there. To satisfy \eqref{eq:d-F-k}, we'd like
\(q_F(k) = q_F(k - 1) + r_F(k)\)
and for \eqref{eq:ell-F-k}
\(r_{F + 1}(k) = 3 q_F(k - 1) + 5 F r_F(k)\).
Since the forward difference \(\Delta q_F(k) = r_F(k + 1)\) is known,
we can inductively solve for \(q_F\) in terms of \(q_F(0)\). But
\(D_{F, n} = 0\) gives \(q_F(0) = 0\).

For example, since we have \(r_1(k) = 3 \binom{k}{0}\) we'll have
\(q_1(k) = 3 \binom{k}{1}\). Once this is known
\begin{equation}
r_2(k) = 3 q_1(k - 1) + 5 r_1(k) = 3 \cdot 3 \binom{k - 1}{1} +
5 \cdot 3 \binom{k}{0} = 9 \binom{k}{1} + 6 \binom{k}{0}.
\end{equation}
If we write these polynomials in the ``falling factorial'' basis of
forward differences, then we can show that
\begin{equation}
r_F(k) = 3^F \binom{k}{F} + \cdots
\end{equation}
which will complete the proof of the first inequality. To see this, first
note that for a polynomial in this basis
\(f(k) = A \binom{k}{d} + B \binom{k}{d - 1} + C \binom{k}{d - 2} +
D \binom{k}{d - 3} + \cdots\) we have
\begin{align}
f(k + 1) &= A \binom{k}{d} + (A + B) \binom{k}{d - 1} +
  (B + C) \binom{k}{d - 2} + (C + D) \binom{k}{d - 3} + \cdots \\
f(k - 1) &= A \binom{k}{d} + (B - A) \binom{k}{d - 1} +
  (C - B + A) \binom{k}{d - 2} + (D - C + B - A) \binom{k}{d - 3} + \cdots
\end{align}
Using these, we can show that if
\(r_F(k) = \sum_{j = 0}^{F - 1} c_j \binom{k}{j}\) then
\begin{align}
q_F(k) &= c_{F - 1} \binom{k}{F} + \sum_{j = 1}^{F - 1}
  (c_j + c_{j - 1}) \binom{k}{j} \\
r_{F + 1}(k) &= 3 \left[-c_0 \binom{k}{0} +
  \sum_{j = 1}^F c_{j - 1} \binom{k}{j}\right] +
5F \left[\sum_{j = 0}^{F - 1} c_j \binom{k}{j}\right] =
3 c_{F - 1} \binom{k}{F} + \cdots
\end{align}
Under the inductive hypothesis \(c_{F - 1} = 3^F\) so that
the lead term in \(r_{F + 1}(k)\) is \(3 c_{F - 1} \binom{k}{F}
= 3^{F + 1} \binom{k}{F}\).

For the second inequality, we'll show that
\begin{equation}
\sum_{k = 0}^{n - 1} \gamma_{3k + 5F} L_{F, k} \leq
  \left[q_{F + 1}(n) \mach^{F + 1} +
  \bigO{\mach^{F + 2}}\right] \cdot \widetilde{p}(s)
\end{equation}
and then we'll have our result since we showed above that
\(q_{F + 1}(n) = 3^{F + 1} \binom{n}{F + 1} + \bigO{n^F}\). Since
\(\gamma_{3k + 5F} L_{F, k} \leq (3k + 5F) L_{F, k} \mach +
\bigO{\mach^{F + 2}} \widetilde{p}(s)\) it's enough to consider
\begin{equation}
\sum_{k = 0}^{n - 1} (3k + 5F) r_F(n - k) =
\sum_{k = 1}^n (3(n - k) + 5F) r_F(k).
\end{equation}
Since \(q_F(k) = q_F(k - 1) + r_F(k)\) and \(q_F(0) = 0\) we have
\(q_{F}(n) = \sum_{k = 1}^n r_{F}(k)\) thus
\begin{equation}
q_{F + 1}(n) = \sum_{k = 1}^n r_{F + 1}(k)
= \sum_{k = 1}^n 3 q_F(k - 1) + 5 F r_F(k)
= \sum_{k = 1}^n 3 \left[\sum_{j = 1}^{k - 1} r_F(j)\right] + 5 F r_F(k).
\end{equation}
Swapping the order of summation and grouping like terms, we have our
result.
\end{proof}

\begin{proof}[Proof of Lemma~\ref{lemma:k-order}]\label{proof:k-order}
As in \eqref{eq:matrix-de-casteljau}, we can express the compensated
de Casteljau algorithm as
\begin{equation}
\db{F}^{(k)} = U_{k + 1} \db{F}^{(k + 1)} + \ell_{F}^{(k)}
\Longrightarrow \db{F}^{(0)} = \sum_{k = 0}^{n - 1}
U_1 \cdots U_k \ell_F^{(k)} = \sum_{k = 0}^{n - 1}
\left[\sum_{j = 0}^k \ell_{F, j}^{(k)} B_{j, k}(s)\right].
\end{equation}
For the inexact equivalent of these things, first note that
\(\widehat{r} = (1 - s)(1 + \delta)\). Due to this,
we put the \(\widehat{r}\) term at the end of each update step to reduce
the amount of round-off:
\begin{align}
  \cdb{F}_j^{(k)} &=
  \widehat{\ell}_{F, j}^{(k)} \oplus
  \left(s \otimes \cdb{F}_{j + 1}^{(k + 1)}\right) \oplus
  \left(\widehat{r} \otimes \cdb{F}_j^{(k + 1)}\right) \\
&= (1 - s) \cdot \cdb{F}_j^{(k + 1)}(1 + \theta_3) +
  s \cdot \cdb{F}_{j + 1}^{(k + 1)}(1 + \theta_3) +
  \widehat{\ell}_{F, j}^{(k)} (1 + \theta_2) \\
\Longrightarrow \cdb{F}^{(k)} &=
  U_{k + 1} \cdb{F}^{(k + 1)}(1 + \theta_3) +
  \widehat{\ell}_{F}^{(k)} (1 + \theta_2) \\
\Longrightarrow \cdb{F}^{(0)} &=
  \sum_{k = 0}^{n - 1}
  U_1 \cdots U_k \widehat{\ell}_F^{(k)} (1 + \theta_{3k + 2})
  = \sum_{k = 0}^{n - 1}
  \left[\sum_{j = 0}^k \widehat{\ell}_{F, j}^{(k)} (1 + \theta_{3k + 2})
    B_{j, k}(s)\right].
\end{align}
Since
\begin{equation}
\db{F + 1}_0^{(0)} = \db{F}_0^{(0)} - \cdb{F}_0^{(0)} = \sum_{k = 0}^{n - 1}
\sum_{j = 0}^k \left(\ell_{F, j}^{(k)} -
\widehat{\ell}_{F, j}^{(k)} (1 + \theta_{3k + 2})\right) B_{j, k}(s)
\end{equation}
it's useful to put a bound on \(\ell_{F, j}^{(k)} -
\widehat{\ell}_{F, j}^{(k)} (1 + \theta_{3k + 2})\). Via
\begin{align}
\widehat{\ell}_{F, j}^{(k)} &= e_1 \oplus \cdots \oplus e_{5F - 2} \oplus
\left(\rho \otimes \cdb{F - 1}_j^{(k + 1)}\right) \\
&= e_1\left(1 + \theta_{5F - 2}\right) + \cdots +
e_{5F - 2}\left(1 + \theta_2\right) +
\rho \cdot \cdb{F - 1}_j^{(k + 1)} \left(1 + \theta_2\right)
\end{align}
we see that
\begin{equation}
\left|\ell_{F, j}^{(k)} -
\widehat{\ell}_{F, j}^{(k)} (1 + \theta_{3k + 2})\right| \leq
\gamma_{3k + 5F} \cdot \widetilde{\ell}_{F, j}^{(k)}
\Longrightarrow
\left|\db{F + 1}_0^{(0)}\right| \leq \sum_{k = 0}^{n - 1}
\gamma_{3k + 5F} \sum_{j = 0}^k \widetilde{\ell}_{F, j}^{(k)} B_{j, k}(s).
\end{equation}
Applying \eqref{eq:L-sum-bound} directly gives
\begin{equation}
\left|\db{F + 1}_0^{(0)}\right| \leq
  \left[\left(3^{F + 1} \binom{n}{F + 1} + \bigO{n^F}\right)
  \mach^{F + 1} + \bigO{\mach^{F + 2}}\right] \cdot \widetilde{p}(s).
\end{equation}
Letting \(K = F + 1\) we have our result.
\end{proof}

\begin{proof}[Proof of Theorem~\ref{thm:k-comp-result}]\label{proof:k-comp-result}
Since
\begin{equation}
\mathtt{CompDeCasteljau}(p, s, K) = \mathtt{SumK}\left(\left[
  \widehat{b}_0^{(0)}, \ldots, \cdb{K - 1}_0^{(0)}\right], K\right),
\end{equation}
applying Theorem~\ref{thm:sum-k} tells us that
\begin{multline}\label{eq:sum-k-applied}
\left|\mathtt{CompDeCasteljau}(p, s, K) - \sum_{F = 0}^{K - 1}
\cdb{F}_0^{(0)}\right| \leq \\
\left(\mach + 3 \gamma_{n - 1}^2\right) \left|\sum_{F = 0}^{K - 1}
\cdb{F}_0^{(0)}\right| +
\gamma_{2n - 2}^K \sum_{F = 0}^{K - 1} \left|\cdb{F}_0^{(0)}\right|.
\end{multline}
Since
\begin{equation}
p(s) = b_0^{(0)} = \widehat{b}_0^{(0)} + \db{1}_0^{(0)}
= \cdots
= \widehat{b}_0^{(0)} + \cdb{1}_0^{(0)} + \cdots
+ \cdb{K - 1}_0^{(0)} + \db{K}_0^{(0)}
\end{equation}
we have
\begin{gather}
\left|\sum_{F = 0}^{K - 1} \cdb{F}_0^{(0)}\right|
\leq \left|p(s)\right| + \left|\db{K}_0^{(0)}\right| \quad \text{and} \\
\begin{multlined}
\left|\mathtt{CompDeCasteljau}(p, s, K) - p(s)\right| \leq \\
\left|\mathtt{CompDeCasteljau}(p, s, K) - \sum_{F = 0}^{K - 1}
\cdb{F}_0^{(0)}\right| +
\left|\db{K}_0^{(0)}\right| \label{eq:triangle-ps}.
\end{multlined}
\end{gather}
Due to Lemma~\ref{lemma:k-order}, \(\db{F}_0^{(0)} =
\bigO{\mach^F} \widetilde{p}(s)\), hence
\begin{align}
\left(\mach + 3 \gamma_{n - 1}^2\right) \left|\sum_{F = 0}^{K - 1}
\cdb{F}_0^{(0)}\right| &\leq
\left[\mach + \bigO{\mach^2}\right] \left|p(s)\right| +
\bigO{\mach^{K + 1}} \widetilde{p}(s) \\
\gamma_{2n - 2}^K \sum_{F = 0}^{K - 1} \left|\cdb{F}_0^{(0)}\right| &\leq
\gamma_{2n - 2}^K \left|\widehat{b}_0^{(0)}\right| +
\bigO{\mach^{K + 1}} \widetilde{p}(s) \\
&\leq
\gamma_{2n - 2}^K \left[\left|p(s)\right| +
  \bigO{\mach} \widetilde{p}(s)\right] +
\bigO{\mach^{K + 1}} \widetilde{p}(s).
\end{align}
Combining this with \eqref{eq:sum-k-applied} and \eqref{eq:triangle-ps}, we
see
\begin{align}
& \left|\mathtt{CompDeCasteljau}(p, s, K) - p(s)\right| \\
\leq &
\left[\mach + \bigO{\mach^2}\right] \left|p(s)\right| +
\left|\db{K}_0^{(0)}\right| +
\bigO{\mach^{K + 1}} \widetilde{p}(s) \\
\leq &
\left[\mach + \bigO{\mach^2}\right] \left|p(s)\right| +
\left[\left(3^{K} \binom{n}{K} + \bigO{n^{K - 1}}\right) \mach^K +
\bigO{\mach^{K + 1}} \right]
\widetilde{p}(s).
\end{align}
Dividing this by \(\left|p(s)\right|\), we have our result.
\end{proof}

\begin{lemma}\label{lemma:bad-triangle}
Consider three smooth curves
\(b_0, b_1, b_2\) that form a closed loop: \(b_0(1) = b_1(0)\),
\(b_1(1) = b_2(0)\) and \(b_2(1) = b_0(0)\).
Take \textbf{any} smooth map \(\varphi(s, t)\) on \(\utri\) that
sends the edges to the three curves:
\begin{equation}
\varphi(r, 0) = b_0(r), \quad \varphi(1 - r, r) = b_1(r),
  \quad \varphi(0, 1 - r) = b_2(r) \quad \text{for } r \in \left[0, 1\right].
\end{equation}
Then we must have
\begin{equation}
2 \int_{\utri} \det(D\varphi) \left[F \circ \varphi\right] \, dt \, ds =
\oint_{b_0 \cup b_1 \cup b_2} H \, dy - V \, dx
\end{equation}
for antiderivatives that satisfy \(H_x = V_y = F\).

When \(\det(D\varphi) > 0\), this is just the change of variables
formula combined with Green's theorem.
\end{lemma}

\begin{proof}
Let \(x(s, t)\) and \(y(s, t)\) be the components of \(\varphi\). Define
\begin{equation}
\Delta S = H(x, y) y_s - V(x, y) x_s \quad \text{and} \quad
\Delta T = H(x, y) y_t - V(x, y) x_t.
\end{equation}
On the unit triangle \(\utri\), Green's theorem gives
\begin{equation}\label{eq:basic-greens}
\int_{\mathcal{U}} \left[\partial_s \Delta T -
  \partial_t \Delta S\right] \, dV =
\oint_{\partial \mathcal{U}} \Delta S \, ds + \Delta T \, dt.
\end{equation}
The boundary \(\partial \utri\) splits into the bottom edge \(E_0\),
hypotenuse \(E_1\) and left edge \(E_2\).

Since
\begin{equation}
E_0 = \left\{ \left[ \begin{array}{c} r \\ 0 \end{array}\right] \mid
  r \in \left[0, 1\right] \right\}
\end{equation}
we take \(\varphi(r, 0) = b_0(r)\) hence
\begin{equation}
dx = x_s \, dr, dy = y_s \, dr \Longrightarrow
H dx - V dy = \Delta S \, dr.
\end{equation}
We also have \(ds = dr\) and \(dt = 0\) due to the
parameterization, thus
\begin{equation}
\int_{E_0} \Delta S \, ds + \Delta T \, dt =
  \int_{r = 0}^{r = 1} \Delta S \, dr = \int_{b_0} H \, dx - V \, dy.
\end{equation}
We can similarly verify that
\(\int_{E_j} \Delta S \, ds + \Delta T \, dt = \int_{b_j} H \, dx - V \, dy\)
for the other two edges. Combining this with~\eqref{eq:basic-greens}
we have
\begin{equation}
\int_{\mathcal{U}} \left[\partial_s \Delta T -
  \partial_t \Delta S\right] \, dV =
\oint_{b_0 \cup b_1 \cup b_2} H \, dx - V \, dy.
\end{equation}
To complete the proof, we need
\begin{equation}
\int_{\mathcal{U}} \left[\partial_s \Delta T -
  \partial_t \Delta S\right] \, dV =
2 \int_{\mathcal{U}} \det(D\varphi) \left[F \circ \varphi\right] \, dV
\end{equation}
but one can show directly that
\begin{equation}
\partial_s \Delta T - \partial_t \Delta S =
  2 \left(x_s y_t - x_t y_s\right) F(x, y) =
  2 \det(D\varphi) \left[F \circ \varphi\right]. \tag*{\qedhere}
\end{equation}
\end{proof}
