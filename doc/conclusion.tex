\chapter{Conclusion}

\section{Solution Transfer}

This work has described a method for conservative interpolation
between curved meshes. The transfer process conserves globally
to machine precision since we can use exact quadratures for all
integrals. The primary source of error comes from solving the linear
system with the mass matrix for the target mesh. This allows
less restrictive usage of mesh adaptivity, which can make computations
more efficient. Additionally, having a global transfer algorithm
allows for remeshing to be done less frequently.

The algorithm breaks down into three core subproblems: B\'{e}zier triangle
intersection, an advancing front for intersecting elements and
integration on curved polygons. The inherently local nature of the
advancing front allows the algorithm to be parallelized via domain
decomposition with little data shared between processes. By
restricting integration to the intersection of elements from the
target and donor meshes, the algorithm can accurately transfer
both continuous and discontinuous fields.

\subsection{Future Work}

As mentioned in the preceding chapters, there are several research
directions possible to build upon the solution transfer algorithm.
The usage of Green's theorem nicely extends to \(\reals^3\) via
Stoke's theorem, but the B\'{e}zier triangle intersection algorithm
is specific to \(\reals^2\). The equivalent B\'{e}zier tetrahedron
intersection algorithm is significantly more challenging.

The restriction to shape functions from the global coordinates basis
is a symptom of the method and not of the inherent problem. The
pre-image basis has several appealing properties, for example
this basis can be precomputed on \(\utri\). The problem of a
valid tessellation of a curved polygon warrants more exploration.
Such a tessellation algorithm would enable usage of the pre-image
basis.

The usage of the global coordinates basis does have some benefits.
In particular, the product of shape functions from different meshes
is still a polynomial in \(\reals^2\). This means that we could
compute the coefficients of \(F = \phi_0 \phi_1\) directly and
use them to evaluate the antiderivatives \(H\) and \(V\) rather
than using the fundamental theorem of calculus. Even if this
did not save any computation, it may still be preferred over
the FTC approach because it would remove the usage of quadrature
points outside of the domain \(\mathcal{P}\).

\section{Ill-conditioned B\'{e}zier Curve Intersection}

The B\'{e}zier triangle intersection subproblem can be solved
partially by intersecting B\'{e}zier curves (i.e. the edges).
However, the probability of ``almost tangent'' or ill-conditioned
B\'{e}zier curve intersections increases to unity as the mesh
size \(h\) goes to \(0\).

Turning our focus to ill-conditioned B\'{e}zier curve intersections,
we designed a modified Newton's method that computed the residual
as if in extended precision. Building on existing work, we describe
a compensated de Casteljau algorithm that allows the evaluation of
a polynomial in Bernstein form as accurate as if the computations
were done in \(K\)-times the working precision. The residual
\(F(s, t) = b_0(s) - b_1(t)\) depends on four evaluations of
a polynomial in Bernstein form, so \(F\) can be computed more precisely
by using the compensated evaluation algorithm.
